
% Patrick Sahle (Göttingen). Digitales Archiv – Digitale Edition. Anmerkungen zur Begriffsklärung

\section{Digital Archives}
\begin{frame}{Digital long-term archiving}
\metroset{block=fill}\footnotesize
\begin{columns}
\column{0.68\textwidth}
\begin{itemize}
\item ensuring the authentic and sustainable availability of digital ressources on the level of the bitstream and on a semantic level
\item integral principle to every form of sustainable data storage
\item begins with the data production in a sustainable data format (and ideally, following a recognized data standard)
\item requires standardization of data formats and archiving workflows
\item serves both the dissemination as well as the preservation of digital content
\item not just about technical solutions but also institutional stability and policies
\end{itemize}
\column{0.28\textwidth}

\begin{block}{A digital archive}
\begin{itemize}
\item is more than a mere collection of scanned book pages or digitized images, etc.
\item if offers metadata, norm data and controlled vocabularies
\end{itemize}
\end{block}

\end{columns}
\end{frame}
%---------------------------------


\begin{frame}{Digital Archives}
\metroset{block=fill}\footnotesize
\begin{columns}
\column{0.55\textwidth}
\begin{block}{Digital Archive}
\begin{itemize}
\item organized collection of digital objects (text, images, audio, video and multimedia streams)
\item digital objects are described by standards both in terms of contents (e.g. TEI) and bibliographically (e.g. Dublin Core)
\item published sustainably using interfaces, services and APIs (e.g. OAI-PMH)
\item digital objects have unique, persistent and citable identifiers (e.g. DOI, URN, PURL, PID)
\item authenticity of objects is checked by means of digital signatures or checksums (`Is the number of bytes in the object still the same it used to be?')
\end{itemize}
\end{block}
\column{0.43\textwidth}
\begin{block}{Trustworthy Digital Archives}
as defined by the Research Libraries Group (RLG)
\begin{itemize}
\item secure organizational structure and legal status
\item financial sustainability
\item technological and procedural aptness
\item ensuring data and system security
\item documentation and transparency
\item conformity with the OAIS standard
\end{itemize}
\end{block}
\end{columns}
$\to$ retro-digitizing objects but also standards for new (born digital) resources.
\end{frame}

%---------------------------------

\begin{frame}{Open Archival Information System Reference Mode}
\metroset{block=fill}
\begin{columns}
\column{0.48\textwidth}
\begin{block}{OAIS}
Generic model for the organization of a digital archive 

$\to$ developed 1995--2002 by the Consultative Committee for Space Data Systems (CCSDS) 
\end{block}
\column{0.48\textwidth}
\begin{block}{Tasks for Digital Archives according to OAIS}
\begin{enumerate}
\item data ingest
\item archival storage
\item data management
\item system administration
\item preservation planning
\item access
\end{enumerate}
\end{block}
\end{columns}
\medskip 

\alert{What's the difference between a Digital Archive and a Digital Scholarly Edition?
\parencite{sahleArchiv2007} 

\faArrowCircleDown}

\end{frame}

%---------------------------------

\begin{frame}[allowframebreaks]{Digital Archives vs. Digital Editions}
\metroset{block=fill}
\footnotesize
\begin{columns}
\column{0.38\textwidth}

\begin{block}{Digital documents}\scriptsize
Material objects are the target of digitization but digitization doesn't reproduce them -- it represents them in a digital format.
A digital document is a view on the original material object.
\end{block}

\begin{block}{Archive}\scriptsize
Traditionally an archive is an ordered collection of documents with the goal of documenting them, preserving them in the long-term and making them accessible. 
$\to$ this function isn't only carried out by actual archives but also by  museums, libraries and other cultural heritage institutions. 
Traditional archive material doesn't need to be represented because its physical objects can be accessed directly. 
\end{block}

(we have already def'd editions)

\column{0.6\textwidth}
\scriptsize
Editions are often based in archival material. The edition isn't a storage device, it is a publication of a historical source and the editorial work done on it. 
While the edition contains lots of editorial work and enrichment, archival documents are usually original and largely unprocessed. A non-digital archive isn't a form of publication in and of itself -- a digital archive \emph{is} a form of publication, too, like a digital edition 
$\to$ blurs the lines a bit. 

We could say that \textbf{the difference lies in the depth of data enrichment and editorial work.} $\to$ once a presentation form is provided, data from a digital archive becomes a digital edition.
% erschließende Wiedergabe. 

% Dahlström, Mats: How Reproductive is a Scholarly Edition? In: Literary and Linguistic Computing 19/1 (2004), S. 17–33.

% Robinson, Peter: What is a Critical Digital Edition? In: Variants. Journal of the European Society for Textual Scholarship 1 (2002), S. 43–62.

\begin{block}{Digital Archive}\scriptsize
A main goal of a digital archive is to preserve and publish a specific choice of documents as a collection (ideally, representative and well-balanced). The digital objects aren't necessarily direct representations but may have undergone editorial intervention (e.g. normalized orthography). 
Its documents should be uniform in how they are encoded and processed.
In the case of the single source principle, the edition is generated dynamically (`on the fly') from the archived source data. (If any edition is provided).
\end{block}

\end{columns}

\end{frame}
%---------------------------------

