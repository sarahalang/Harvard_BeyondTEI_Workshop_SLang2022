
\section{Motivation}

\begin{frame}{TEI, now what?}
\metroset{block=fill}
    Why are we doing this workshop? The motivation from our abstract:
    \begin{itemize}
        \item \punkti the Text Encoding Initiative (TEI) for XML has become the gold standard for scholarly editions of texts.
        \item But what happens after an edition is encoded in TEI? 
        \item While it is an \textbf{ideal format for archiving digital data}, it is \alert{less than ideal for viewing and interacting with the edited text.}
    \end{itemize}
    
    \begin{block}{Goals for the next session}
    \begin{enumerate}
        \item understand the terms from the abstract:
        \begin{itemize}
            \item[\textcolor{alert}{\faClose}] digital edition
            \item[\textcolor{alert}{\faClose}] single source principle
            \item[\textcolor{alert}{\faClose}] \sout{creating different representations from our data} $\to$ (why) do we even need to create new presentations from our data? Aren't there tools for that?
        \end{itemize}
    \end{enumerate}
    \end{block}
\end{frame}

%--------------------------------------------

\begin{frame}[allowframebreaks]{Teaser}

    \href{http://gams.uni-graz.at/o:ufbas.1563\#Eintrag-149}{GAMS ufbas}

    \includegraphics[width=0.48\textwidth]{img/ufbas1.png}
    \includegraphics[width=0.48\textwidth]{img/ufbas2.png}
    
    \framebreak
    
    \href{https://gams.uni-graz.at/o:corema.pa1.recipes}{CoReMA} $\to$
    Look at: \protect\url{http://gams.uni-graz.at/corema}
    
    \includegraphics[width=0.6\textwidth]{img/corema-b1.png}
    
    \framebreak
    
    \href{https://furnaceandfugue.org}{Furnace and Fugue} (music in MEI, text and images)
    \begin{columns}
    \column{0.63\textwidth}
    \includegraphics[width=\textwidth]{img/fnf1.png}
    \column{0.33\textwidth}
    \includegraphics[width=\textwidth]{img/fnf2.png}
    \end{columns}
    
    
\end{frame}

%--------------------------------------------

\begin{frame}{From TEI to Edition}
What did all those examples have in common?
    \begin{itemize}\footnotesize
        \item all are digital editions
        \item all are based on data in TEI-XML
    \end{itemize}

That's what we want. -- How do we get there?

\metroset{block=fill}
\begin{block}{Remember}
\begin{itemize}\footnotesize
    \item XML is great for capturing \& long-term archiving data in its whole complexity
    \begin{itemize}
        \item[$\to$] \dots but also ends up chaotic with a lot of detail
        \item[$\to$] theoretically human-readable
        \item[$\to$] in practice, it's better to create a new representation of the data to read
    \end{itemize}
    \item \textbf{Desired result: }view/presentation, visualization and interaction with the data as a website (i.e. HTML data)
    \item \alert{$\to$ we need to transform the data}
\end{itemize}
\end{block}

\end{frame}

%--------------------------------------------


\begin{frame}{TEI Publication Tools}
There is a plethora of out-of-the-box solutions to make editions out of TEI data. 
$\to$
\alert{\href{https://wiki.tei-c.org/index.php/Category:Publishing_and_delivery_tools}{TEI List of Resources for Publishing and Delivery Tools}}, for example:
\begin{enumerate}
    \item teiPublisher
    \item EVT
    \item TEICHI
    \item Versioning Machine
    \item Boilerplate
    \item Oxgarage
    \item GAMS \dots 
\end{enumerate}

\end{frame}

%--------------------------------------------

\begin{frame}[allowframebreaks]{TEI Publisher: The Instant Publishing Toolbox}
    \begin{itemize}
        \item \protect\url{https://teipublisher.com/index.html}
        \item \protect\url{https://wiki.tei-c.org/index.php/TEI_Publisher}
        \item \protect\url{https://wiki.tei-c.org/index.php/TeiPublisher}
        \item \protect\url{http://hisoma.huma-num.fr/exist/apps/tei-publisher/index.html}
        \item ``Testimonial'': \href{https://digitalintellectuals.hypotheses.org/3912}{Florian Chiffoleau (2020), Blog Post: \emph{Publication of my digital edition -- Working with TEI Publisher.}}
    \end{itemize}
    
    \includegraphics[width=0.7\textwidth]{img/tei-publisher1.png}
    
    \framebreak
    
    Demos:  \href{https://teipublisher.com/exist/apps/vangogh/let001.xml}{Van Gogh Letters} \sep \href{https://teipublisher.com/exist/apps/eebo/index.html}{Early English Books Online (EEBO)}
    
    \begin{columns}
    \column{0.33\textwidth}
    \includegraphics[width=\textwidth]{img/tei-publisher2.png}
    \column{0.33\textwidth}
    \includegraphics[width=\textwidth]{img/tei-publisher3.png}
    \column{0.33\textwidth}
    \includegraphics[width=\textwidth]{img/tei-publisher4.png}
    \end{columns}
\end{frame}


%--------------------------------------------

\begin{frame}{Edition Visualization Technology (EVT)}

\begin{columns}
\column{0.53\textwidth}
\begin{itemize}\scriptsize
    \item \protect\url{http://evt.labcd.unipi.it/} -- \protect\url{http://evt-project.sourceforge.net/}
    \item \protect\url{https://visualizationtechnology.wordpress.com/}
    \item Roberto Rosselli Del Turco, Giancarlo Buomprisco, Chiara Di Pietro, Julia Kenny, Raffaele Masotti and Jacopo Pugliese: \emph{Edition Visualization Technology: A Simple Tool to Visualize TEI-based Digital Editions}, Journal of the Text Encoding Initiative 8 (2014/15): Selected Papers from the 2013 TEI Conference “TEI Processing: Workflows and Tools”; \protect\url{https://doi.org/10.4000/jtei.1077}
    \item Del Turco, Roberto Rosselli. “Designing an Advanced Software Tool for Digital Scholarly Editions: The Inception and Development of EVT (Edition Visualization Technology).” \emph{Textual Cultures}, vol. 12, no. 2, 2019, pp. 91–111. JSTOR, \protect\url{www.jstor.org/stable/26821538}. 
\end{itemize}
\column{0.45\textwidth}
\includegraphics[width=\textwidth]{img/evt-example.png}
\end{columns}



\end{frame}

%--------------------------------------------

\begin{frame}{TeiCHI – Bringing TEI Lite to Drupal}


\begin{columns}
\column{0.38\textwidth}
\begin{itemize}\scriptsize
    \item TEIChi: a TEI lite integration into Drupal (\protect\url{http://teichi.org})
    \item Sebastian Pape, Christof Schöch and Lutz Wegner: \emph{TEICHI and the Tools Paradox. Developing a Publishing Framework for Digital Editions}, Journal of the Text Encoding Initiative 2 (2012): Selected Papers from the 2010 TEI Conference; \protect\url{https://doi.org/10.4000/jtei.432}
    \item \href{https://wiki.tei-c.org/index.php/TEICHI}{TEICHI wiki}
\end{itemize}
\column{0.68\textwidth}
\includegraphics[width=\textwidth]{img/teichi-example.png}
\end{columns}



\end{frame}

%--------------------------------------------

\begin{frame}{The Versioning Machine}

\begin{itemize}\small
    \item Versioning Machine (\protect\url{http://v-machine.org/})
    \item \href{https://tei-c.org/activities/projects/the-versioning-machine}{The TEI Versioning Machine}
    \href{https://digitalhumanities.duke.edu/tools/versioning-machine}{Versioning Machine (Duke)}
\end{itemize}

\includegraphics[width=0.9\textwidth]{img/tei-versioning-machine.png}

\end{frame}

%--------------------------------------------

\begin{frame}{TEI Boilerplate}

\begin{columns}
\column{0.48\textwidth}
\begin{itemize}\footnotesize
    \item \protect\url{http://teiboilerplate.org/}
    \item Boilerplate (\protect\url{http://dcl.slis.indiana.edu/teibp/})
    \item \href{http://dcl.slis.indiana.edu/teibp/content/demo.xml}{Demo}
    \item \href{https://wiki.tei-c.org/index.php/TEI_Boilerplate}{TEI Boilerplate Wiki}
    \item \href{https://www.i-d-e.de/wp-content/uploads/2014/07/boilerplate.pdf}{Slides on TEI Boilerplate}
    \item \href{http://tei.it.ox.ac.uk/Talks/2015-07-dhoxss/ex-boilerplate.pdf}{Exercise on TEI Boilerplate}
    \item \href{http://journalofdigitalhumanities.org/2-3/tei-boilerplate}{Poster on TEI Boilerplate}
\end{itemize}

\column{0.48\textwidth}
\includegraphics[width=\textwidth]{img/tei-boilerplate-example.png}
\end{columns}

\end{frame}

%--------------------------------------------

\begin{frame}{Oxgarage}
\metroset{block=fill}\footnotesize

\begin{columns}
\column{0.48\textwidth}
\begin{itemize}
    \item REST-based transformation web service for TEI documents (\protect\url{https://oxgarage.tei-c.org/})
    \item allows you to transform a number of (markup-based) formats to TEI or create them from TEI (\texttt{.html}, \texttt{.tex}, \texttt{.docx}, \dots)
    \item \href{https://tei-c.org/Vault/P5/3.6.0/doc/tei-xsl}{Based on TEI base stylesheets} (\protect\url{http://www.tei-c.org/Tools/Stylesheets/})
    \item[\textcolor{w3schools}{\faCheck}] very useful tool
    \item[\textcolor{alert}{\faClose}] not all TEI elements taken into account, not customizable
\end{itemize}
\column{0.48\textwidth}
\begin{block}{Oxgarage}
\dots offers a set of standard stylesheets to convert between TEI-encoded XML documents and other documents, mostly markup-based, such as \texttt{.docx}, \texttt{.html} or \texttt{.tex}. 

However, not all elements are covered and you have no control over how exactly they are processed. 
\end{block}
\end{columns}\medskip

Tools make things easier superficially but can come at a cost: potential bugs, less than ideal usability, lack of control and customizability:

\alert{$\to$ to customize we need to write our own transformation using XSL stylesheets}


\end{frame}



%--------------------------------------------

\begin{frame}[allowframebreaks]{GAMS (Humanities Asset Management System)}
\metroset{block=fill}
    \begin{columns}
    \column{0.44\textwidth}
    \begin{itemize}\small
        \item GAMS (\emph{Geisteswissenschaftliches Asset Management System} = AMS for the Humanities)
        \item based on \textbf{FEDORA} (\emph{Flexible Extensible Digital Object Repository Architecture}) = infrastructure dedicated to the persistent archival and management of resources considered to be worthy of long-term preservation.
    \end{itemize}
    \column{0.55\textwidth}
    \includegraphics[width=\textwidth]{img/gams.png}
    \end{columns}
    \bigskip 
    
    \protect\url{https://gams.uni-graz.at}
    
    \framebreak
    
    \begin{columns}
    \column{0.58\textwidth}
    \begin{itemize}\small
        \item user access provided through the \textbf{Cirilo client}
        \item \textbf{functionalities:} object creation and management, versioning, normalization \& standards, choice of data formats.
        \item offers a plethora of \textbf{pre-defined content models} for data such as TEI, MEI, LIDO, SKOS, ontologies, R code and story lines
        \item $\to$ offers publication pipelines but is highly customizable
        \item more info: \protect\url{http://gams.uni-graz.at/doku}
    \end{itemize}
    \column{0.48\textwidth}
    \includegraphics[width=\textwidth]{img/gams.png}
    \end{columns}
\end{frame}



%--------------------------------------------

\begin{frame}[allowframebreaks]{Long-term archiving heritage data}
\metroset{block=fill}
    \begin{columns}
    \column{0.4\textwidth}
    \begin{block}{Long-term preservation} denotes the process of maintaining, curating and keeping data usable over a long period of time (10+~years).
    \end{block}
    \column{0.58\textwidth}
    \begin{block}{Key functionalities of long-term archiving architectures include}
    \begin{itemize}\footnotesize
        \item persistent identification, 
        \item versioning, 
        \item support of different data formats, 
        \item management of associated metadata, 
        \item data export and retrieval, 
        \item security and scalability. 
    \end{itemize}
    
    Special emphasis is placed on 
    \begin{itemize}\footnotesize
        \item sustainability,
        \item citability \& 
        \item guarantee of long-term access to the contained resources. 
    \end{itemize}
    \end{block}
    \end{columns}
    
    \begin{block}{XML is great for long-term archiving!}
        Consequently, data formats and software used for preservation should follow \textbf{open source \& non-proprietary standards}; data is ideally encoded in an \textbf{unicode XML format}:
        \begin{itemize}
            \item plain text files are small
            \item human- \& machine-readable
            \item recognized standard stable since 1998
            \item more on XML later\dots
        \end{itemize}
    \end{block}
    
 \end{frame}



%--------------------------------------------

\begin{frame}[allowframebreaks]{Examples of data quality in GAMS}   
\metroset{block=fill}
    \begin{columns}
    \column{0.56\textwidth}
    \includegraphics[width=\textwidth]{img/ufbas1.png}
    \includegraphics[width=\textwidth]{img/gams-graf-dc-metadata.png}
    \column{0.4\textwidth}
    \small
     $\leftarrow$ {\footnotesize \protect\url{http://gams.uni-graz.at/o:ufbas.1563}}
    \bigskip
    
    \begin{itemize}
        \item \textbf{top image:} info on recommended citation, downloadable source data in XML \& RDF
        \item \textbf{below:} `data view' of the Dublin Core (DC) metadata for an XML object in GAMS (ensuring citability, etc.)
    \end{itemize}
    \bigskip 
    
    $\leftarrow$ {\scriptsize \protect\url{http://gams.uni-graz.at/archive/objects/o:graf.2387/methods/sdef:Object/getDC?}}
    \end{columns}
    
    \begin{itemize}
        \item GAMS isn't an out-of-the-box tool -- it's something between a \textbf{Content Management System (CMS)}, a \textbf{repository} (long-term archiving) and a \textbf{publication platform}. 
        \item The XSLT transformations applied to files in GAMS are custom but there are \textbf{wippets} (standard javascript functionalities $\to$ \emph{widget + snippet}, \href{https://github.com/KONDE-AT/gams-wippets}{example here}) and \textbf{standard templates} which can be reused. 
        \item as a publication platform: presentation and long-term archiving, allowing for persistent access to the data, versioning, etc.
    \end{itemize}
    
    One more example: \protect\url{https://gams.uni-graz.at/beurb}
    \bigskip
    
    \begin{columns}
    \column{0.5\textwidth}
    
   \begin{block}{Metadata view on XML data}
    \includegraphics[width=\textwidth]{img/gams-beurb1.png}
   \end{block}
    \column{0.5\textwidth}
    \begin{block}{Map view linked to edited text} 
    \includegraphics[width=\textwidth]{img/gams-beurb2.png}
    \end{block}
    \end{columns}
    Both views are created `on-the-fly' from the same XML file (=\emph{single source principle}, more on that later\dots).
\end{frame}


%-----------------------------------------------------



\begin{frame}[standout]

  \alert{Practice!}
  
  \normalsize
  Use \alert{\protect\url{https://oxgarage.tei-c.org/}} to transform your data to HTML. What do you notice? Do you like it or would you need to customize?
\end{frame}



