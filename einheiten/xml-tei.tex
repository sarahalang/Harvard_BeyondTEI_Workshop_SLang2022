

\section{Annotating with XML markup}

\begin{frame}{TEI, now what?}
\metroset{block=fill}
    Why are we doing this workshop? The motivation from our abstract:
    \begin{itemize}
        \item \punkti the Text Encoding Initiative (TEI) for XML has become the gold standard for scholarly editions of texts.
        \item \dots
    \end{itemize}
    
    \begin{block}{Goals for the next session}
    \begin{enumerate}
        \item wait, what was\dots
        \begin{itemize}
            \item[\textcolor{alert}{\faClose}] XML?
            \item[\textcolor{alert}{\faClose}] TEI?
            \item[\textcolor{alert}{\faClose}] How do I use the TEI for digital editing?
        \end{itemize}
    \end{enumerate}

    \end{block}
\end{frame}

%-----------------------------------------------------
\begin{frame}{XML: eXtensible Markup Language}
\begin{columns}
\column{0.35\textwidth}
\begin{itemize}\small 
    \item \href{https://www.w3schools.com/xml/default.asp}{W3Schools Tutorial} 
    \item {paradigm of the separation of form and content} 
    \item {XML is a metalanguage}
\end{itemize}
\bgupper{w3schools}{black}{.xml} \\

\begin{itemize}\scriptsize 
    \item {RSS}, SOAP, XAML 
    \item {MathML}, {GraphML}~ 
    \item {XHTML}~
    \item {RDF}~
    \item {KML}~ 
    \item {Scalable Vector Graphics (SVG)}
\end{itemize}

\column{0.65\textwidth}
\metroset{block=fill}
\begin{block}{}
\begin{quote}
    \textbf{Extensible Markup Language (XML)} is a \textbf{markup language} and file format for storing, transmitting, and reconstructing arbitrary data. It defines a \textbf{set of rules for encoding documents} in a format that is \textbf{both human-readable and machine-readable.}  (\href{https://en.wikipedia.org/wiki/XML}{Wikipedia})
\end{quote}
\end{block}
\end{columns}

\end{frame}


%----------------------------------

\begin{frame}[fragile]{XML rules}
\begin{columns}
\column{0.42\textwidth}
\small
XML can be checked for \textbf{validity} (validation if it complies with a standard) and \textbf{well-formedness} (following the rules of XML) $\to$ will only be parsed if well-formed. Thus: \alert{Heed thy error messages!}\smallskip

There are rules on how elements can be named (you can look them up if relevant or will get informed by an error message). 

\bigskip

%\includegraphics[width=0.1\textwidth]{doppelkeks.jpg}~\bg{alert}{white}{Doppelkeks} ~ ~\bg{alert}{white}{russische Puppe}~\includegraphics[width=0.15\textwidth]{matroschka.jpg}\vspace{1em}

\mycommand{<key>value</key>}{XML as a key value notation} 
\column{0.55\textwidth}\footnotesize
\metroset{block=fill}
\begin{block}{Rules}
\begin{itemize}
    \item Hierarchical nesting below the root
    \item exactly one root element, i.e. one out-most russion doll
    \item start and end tag 
    \item tag names are case-sensitive (!) 
    \item empty elements allowed (\& can be shortened) 
\end{itemize}
\end{block}
\bigskip 

\begin{block}{Minimal example}
\begin{xmlcode}
<?xml version="1.0" ?>
<root>
  <element attribute="value">
    content
  </element>
  <!-- comment -->
</root>
\end{xmlcode}
\end{block}
\end{columns}

\end{frame}


%----------------------------------


\begin{frame}[fragile,allowframebreaks]{XML rules}

\small

\bg{w3schools}{white}{Prolog}~ \\
\mycommand{<xml version="1.0" encoding="utf-8">}{XML declaration}
\mycommand{<?xsl-stylesheet type="text/xsl" href="my.xsl"?>}{processing instructions  (optional)}
\bigskip

you can include document models (optional) \\
DTD, XML Schema, RelaxNG, Schematron 
\bigskip

\bg{w3schools}{white}{entities}~ `protected' characters that have a meta meaning in XML like: \\
\mycommand{&lt;}{<}
\mycommand{&gt;}{>}
\mycommand{&amp;}{\&}


\end{frame}


%-----------------
\begin{frame}{XML family and vocabularies}
\begin{columns}
\column{0.45\textwidth}
\footnotesize
\bg{w3schools}{white}{XML}~structured description of data \\
\bg{w3schools}{white}{XPath}~navigating xml documents \\
\bg{w3schools}{white}{XML Schema}~strict data model \\
\bg{w3schools}{white}{XSL}~Extensible Stylesheet Language  \\
\bg{w3schools}{white}{XSLT}~XSL-Transformations, i.e. transforming XML documents  \\
\bg{w3schools}{white}{XSL-FO}~ formatted output (e.g. print) \\
\bg{w3schools}{white}{XQuery}~query language for XML databases \\
\bg{w3schools}{white}{and more}~
\column{0.45\textwidth}
\metroset{block=fill}
\begin{block}{}
\footnotesize
\begin{itemize}
    \item \textbf{(X)HTML} Hypertext Markup Language 
    \item \textbf{EAD} Encoded Archival Description 
    \item \textbf{TEI} Text Encoding Initiative 
    \item \textbf{CEI} Charters Encoding Initiative 
    \item \textbf{MEI} Music Encoding Initiative 
    \item \textbf{LIDO} Lightweight Information Describing Objects (describing museum or collection objects)
    \item \textbf{SVG} Scalable Vector Graphics 
    \item \textbf{KML} Keyhole Markup Language (geography)
    \item \textbf{MathML} 
    \item \textbf{CML} Chemical Markup Language, \dots
\end{itemize}
\end{block}
\end{columns}


\end{frame}




\section{Text Encoding Initiative}
%-----------------------------------------------------
\begin{frame}[fragile]{TEI Primer}
\footnotesize\metroset{block=fill}

\begin{columns}
\column{0.48\textwidth}
\bg{alert}{white}{Text Encoding Initiative}\\
\bgupper{w3schools}{black}{.xml}\\
XML-Standard, i.e. convention on how to use XML so that resulting data will be interoperable between different projects.

(founded in 1987, consortium since 2000)

\begin{block}{}
\begin{quote}
    The Text Encoding Initiative (TEI) is a text-centric community of practice in the academic field of digital humanities, operating continuously since the 1980s. The community currently runs a mailing list, meetings and conference series, and maintains the TEI technical standard, a journal, a wiki, a GitHub repository and a toolchain. (\href{https://en.wikipedia.org/wiki/Text_Encoding_Initiative}{Wikipedia})
\end{quote}
\end{block}

\column{0.48\textwidth}
\begin{block}{TEI minimal example}
\begin{xmlcode}
<TEI> <!-- root element -->
    <teiHeader> 
      <!-- author, title, dating, 
           sources, edition rules, etc. -->
    </teiHeader> 
    <text> ... </text>
</TEI>
\end{xmlcode}
\end{block}

\begin{block}{Resources}
    \begin{itemize}\scriptsize
        \item \href{http://www.tei-c.org/Support/Learn/}{Learn TEI} 
        \item \href{http://www.tei-c.org/support/learn/teach-yourself-tei/}{Teach Yourself} \item P5 = 5. Proposal 
        \item MEI for music 
        \item CEI for charters 
        \item \href{http://www.tei-c.org/}{http://www.tei-c.org/} 
    \end{itemize}
\end{block}
\end{columns}

\end{frame}

%-------------------------------------
\begin{frame}[fragile]{TEI Header}
\footnotesize
\bg{alert}{white}{fileDesc}~ = bibliographical description of the contents of the document  \\
\bg{alert}{white}{encodingDesc}~ = connection of electronic document to source (i.e. transcription rules, etc.)  \\

\begin{xmlcode}
<TEI> <!-- root element -->
    <teiHeader>
        <fileDesc> ... </fileDesc> <!-- obligatory --> 
        <encodingDesc> <!-- optional -->
        <profileDesc> <!-- optional -->
        <revisionDesc> <!-- optional -->
    </teiHeader> 
    <text> ... </text>
</TEI>
\end{xmlcode}

\bg{alert}{white}{profileDesc}~ = decribes all non-bibliogaphical aspects of the text (i.e. creation, languages) \\
\bg{alert}{white}{revisionDesc}~ = tracks changes in the digital document 
\end{frame}

%-----------------------------------
\begin{frame}[fragile,allowframebreaks]{Using TEI}
\footnotesize
\href{http://www.tei-c.org/release/doc/tei-p5-doc/en/html/SG.html}{Gentle Intro to XML}

\begin{multicols}{2}
\bg{alert}{white}{TEI Core}~ 
\begin{itemize}
    \item \textbf{div} (division) \item \textbf{p} (paragraph) \item \textbf{head} (heading) \item \textbf{lb} (linebreak) \item \textbf{pb} (page break / beginning) \item \textbf{hi} (highlight) \item \textbf{l} (line) \item \textbf{lg} (line group) \item \textbf{list} \item \textbf{item} \item \textbf{listBibl} \item \textbf{bibl} (bibliographical information)
\end{itemize}

\bg{alert}{white}{Attributes}~ 
\begin{itemize}
    \item \textbf{@n} (label) \item \textbf{@type} (typing) \item \textbf{xml:id} (unique identifier) \item \textbf{xml:lang} (language) \item \textbf{@rend} (rendering) \item \textbf{@ana} (interpretation)
\end{itemize}
\end{multicols}

\begin{xmlcode}
<foreign xml:lang="en">word</foreign>
<term type="homonym"/>
<date when="2009-04-27"/>
<time when="12:00:00"/>
<name type="person"/>
<persName n="Caesar" xml:id="#44BC">Caesaris</persName> 
<!-- or -->
<persName key="ID.01.208"/>
<person/>
<emph/> <hi rend="italic">italic text</hi>
<seg/> <abbr type="acronym"/>
<placeName xml:id="#Whitby">Abbey</placeName>
\end{xmlcode}

\framebreak

\bg{alert}{white}{Name spaces}~ identified via URI 

\bg{w3schools}{white}{<prefix:name>}~
e.g. \texttt{<tei:p>} (`I mean the \texttt{<p>} according to the TEI standard.') \\

\bg{alert}{white}{declaration}~ <element xmlns=“URI“> \dots \\
<prefix:element xmlns:prefix=“URI“> \dots \\
e.g. 
\begin{xmlcode}
<tei:p xmlns:tei=“http://www.tei-c.org/ns/1.0“>...
\end{xmlcode}


\end{frame}



\begin{frame}[fragile]{TEI is organized in modules}
\footnotesize
Acts of speech (\href{http://www.tei-c.org/release/doc/tei-p5-doc/en/html/examples-sp.html}{reference}) if speaker name is mentioned, otherwise  \href{http://www.tei-c.org/release/doc/tei-p5-doc/en/html/examples-said.html}{TEIs `said'}:
\begin{xmlcode}
<sp who="#person">
    <speaker>1.</speaker> <p>Bla, bla, bla.</p>
</sp>

<said who="#Adolphe">- Alors, Albert, quoi de neuf?</said>
\end{xmlcode}

Letters in TEI \href{http://www.tei-c.org/release/doc/tei-p5-doc/en/html/DS.html#DSOC}{(reference)}

\begin{xmlcode}
<div type="letter" n="14">
    <head>Letter XIV: Miss Clarissa Harlowe to Miss Howe</head>
        <opener>
            <dateline>Thursday evening, March 2.</dateline>
            <salute>Hallo,</salute>
        </opener>
    <p>On Hannah's depositing my long letter ...</p>
    <closer>
        <salute>Yours more than my own,</salute>
        <signed>Clarissa Harlowe</signed>
    </closer>
</div>
\end{xmlcode}
\end{frame}

%--------------------------------
\begin{frame}[fragile]{Names, Dates, Places}
\footnotesize
\bg{w3schools}{white}{\emph{Named Entities} \& indirect reference}\\
\href{http://www.tei-c.org/release/doc/tei-p5-doc/en/html/ND.html}{TEI 13: Names, Dates, People, Places} 

\metroset{block=fill}
\begin{columns}
\column{0.4\textwidth}
\begin{itemize}
    \item \textbf{persName} for personal names, \textbf{<rs>} for \emph{referring string} when mentioned indirectly (`he', `the woman', etc.) $\to $\textbf{@key} or \textbf{@ref} to specify who it is  (\href{http://www.tei-c.org/release/doc/tei-p5-doc/en/html/CO.html#CONARS}{reference}). \item \textbf{forename} 
    \item \textbf{surname} 
    \item \textbf{roleName} (z.B. `king') 
    \item  \textbf{genName} (`the Younger') 
    \item \textbf{addName} 
    \item \textbf{nameLink} (`von').
\end{itemize}

\column{0.58\textwidth}
\begin{xmlcode}
<name role="writer" type="person"
ref="http://d-nb.info/gnd/118540238">
Goethe</name>
<person>
  <addName type="Former">Murray</addName>
  <forename>Wilhelmina</forename>
  <addName type="nickname">Mina</addName>
</person>
\end{xmlcode}
\end{columns}

\end{frame}




%-----------------------------------------------------
\begin{frame}[fragile,allowframebreaks]{Metadata in the TEI Header}
\subsection{teiHeader, msDesc, sourceDesc, titlePage}
\small

\begin{xmlcode}
<teiHeader>
 <fileDesc>
  <titleStmt>
   <title>
<!-- title of the resource -->
   </title>
  </titleStmt>
  <publicationStmt>
   <p>
<!-- Information about distribution of the resource -->
   </p>
  </publicationStmt>
  <sourceDesc>
   <p>
<!-- Information about source from which the resource derives -->
   </p>
  </sourceDesc>
 </fileDesc>
</teiHeader>
\end{xmlcode}


The title and author in the \texttt{<titleStmt>} isn't the bibliographic data from the source! It describes the digital document and its authors or editors.

If you want to desribe your source documents, you need elements like \texttt{<sourceDesc>} or \texttt{<msDesc>}:
\begin{xmlcode}
<sourceDesc>
 <bibl>
  <title level="a">The Interesting story of the Children in the Wood</title>. In
 <author>Victor E Neuberg</author>, <title>The Penny Histories</title>.
 <publisher>OUP</publisher>
  <date>1968</date>. </bibl>
</sourceDesc>
\end{xmlcode}

\begin{xmlcode}
<sourceDesc>
 <p>Born digital: no previous source exists.</p>
</sourceDesc>
\end{xmlcode}

\begin{xmlcode}
<teiHeader>
 <fileDesc>
  <titleStmt>
   <title>Thomas Paine: Common sense, a
       machine-readable transcript</title>
   <respStmt>
    <resp>compiled by</resp>
    <name>Jon K Adams</name>
   </respStmt>
  </titleStmt>
  <publicationStmt>
   <distributor>Oxford Text Archive</distributor>
  </publicationStmt>
  <sourceDesc>
   <bibl>The complete writings of Thomas Paine, collected and edited
       by Phillip S. Foner (New York, Citadel Press, 1945)</bibl>
  </sourceDesc>
 </fileDesc>
</teiHeader>
\end{xmlcode}

\end{frame}

%-----------------------------------------------------
\begin{frame}[fragile]{\texttt{<msDesc>}}

\begin{xmlcode}
<msDesc>
 <msIdentifier>
  <settlement>Oxford</settlement>
  <repository>Bodleian Library</repository>
  <idno type="Bod">MS Poet. Rawl. D. 169.</idno>
 </msIdentifier>
 <msContents>
  <msItem>
   <author>Geoffrey Chaucer</author>
   <title>The Canterbury Tales</title>
  </msItem>
 </msContents>
 <physDesc>
  <objectDesc>
   <p>A parchment codex of 136 folios, measuring approx
       28 by 19 inches, and containing 24 quires.</p>
   <p>The pages are margined and ruled throughout.</p>
   <p>Four hands have been identified in the manuscript: the first 44
       folios being written in two cursive anglicana scripts, while the
       remainder is for the most part in a mixed secretary hand.</p>
  </objectDesc>
 </physDesc>
</msDesc>
\end{xmlcode}

$\to$ Use websearch (`tei msDesc') to learn how to use new elements (\href{https://tei-c.org/release/doc/tei-p5-doc/en/html/ref-msDesc.html}{overview} and find \href{https://tei-c.org/release/doc/tei-p5-doc/en/html/examples-msDesc.html}{examples like this}).

\end{frame}




%-----------------------------------------------------
\begin{frame}[fragile]{\texttt{<titlePage>}}

To describe a title page (e.g. early modern print copperplates, etc.), use \texttt{<titlePage>}:
\begin{xmlcode}
<titlePage>
 <docTitle>
  <titlePart type="main">THOMAS OF Reading.</titlePart>
  <titlePart type="alt">OR, The sixe worthy yeomen of the West.</titlePart>
 </docTitle>
 <docEdition>Now the fourth time corrected and enlarged</docEdition>
 <byline>By T.D.</byline>
 <figure>
  <head>TP</head>
  <p>Thou shalt labor till thou returne to duste</p>
  <figDesc>Printers Ornament used by TP</figDesc>
 </figure>
 <docImprint>Printed at <name type="place">London</name> for <name>T.P.</name>
  <date>1612.</date>
 </docImprint>
</titlePage>
\end{xmlcode}

\end{frame}




%-----------------------------------------------------
\begin{frame}[fragile]{\texttt{<front>}}

You might also need \texttt{<front>} (front matter): contains any prefatory matter (headers, abstracts, title page, prefaces, dedications, etc.) found at the start of a document, before the main body. 

\begin{xmlcode}
<front>
 <epigraph>
  <quote>Nam Sibyllam quidem Cumis ego ipse oculis meis vidi in ampulla
     pendere, et cum illi pueri dicerent: <q xml:lang="grc">Σίβυλλα τί
       θέλεις</q>; respondebat illa: <q xml:lang="grc">ὰποθανεῖν θέλω.</q>
  </quote>
 </epigraph>
 <div type="dedication">
  <p>For Ezra Pound <q xml:lang="it">il miglior fabbro.</q>
  </p>
 </div>
</front>
\end{xmlcode}

\end{frame}



%-----------------------------------------------------
\begin{frame}[allowframebreaks]{Making the TEI your own}
\small
\metroset{block=fill}

\begin{columns}
\column{0.5\textwidth}
    \begin{block}{How to find information on TEI elements}
    \dots and teach yourself how to use new elements:
\begin{itemize}\footnotesize
    \item General TEI guidelines (\href{https://tei-c.org/release/doc/tei-p5-doc/en/html/SG.html}{XML Primer}, \href{https://tei-c.org/support/learn/}{Learn the TEI page}, etc.)
    \item web-search TEI + (element you want to know about), i.e. ``tei teiHeader'' and you will get:
    \begin{enumerate}\scriptsize
        \item \href{https://www.tei-c.org/release/doc/tei-p5-doc/en/html/ref-teiHeader.html}{definition page}
        \item \href{https://www.tei-c.org/release/doc/tei-p5-doc/en/html/examples-teiHeader.html}{list of all examples for that element} $\to$ directly over websearch or click `show all' in the examples on the `definitons page'
        \item sometimes even an \href{https://www.tei-c.org/release/doc/tei-p5-doc/en/html/HD.html}{module overview text for things as big as \texttt{<teiHeader>}} (has its own module)
    \end{enumerate}
\end{itemize}
    \end{block}


\column{0.45\textwidth}

\begin{block}{Relevant TEI modules}
\begin{description}\scriptsize
    \item[all] \href{https://tei-c.org/release/doc/tei-p5-doc/en/html/index.html}{All modules}
     \item[5] \href{https://tei-c.org/release/doc/tei-p5-doc/en/html/WD.html}{Characters, Glyphs, and Writing Modes}, 
     \item[10] \href{https://www.tei-c.org/release/doc/tei-p5-doc/en/html/MS.html}{Manuscript Description},
     \item[11] \href{https://tei-c.org/release/doc/tei-p5-doc/en/html/PH.html}{Representation of Primary Sources},
     \item[12] \href{https://tei-c.org/release/doc/tei-p5-doc/en/html/TC.html}{Critical Apparatus},
     \item[13] \href{https://tei-c.org/release/doc/tei-p5-doc/en/html/ND.html}{Names, Dates, People, and Places}.
\end{description}
\end{block}

\begin{block}{}
\footnotesize
Also: The TEI guidelines are documentation and reference, not necessarily ideal teaching tools $\to$ overwhelming. Maybe try other tutorials like the \href{https://teibyexample.org/tutorials/TBED00v00.htm}{TEI by example page}, for example the tutorials on \alert{\href{https://teibyexample.org/tutorials/TBED06v00.htm}{Primary Sources}} and \alert{\href{https://teibyexample.org/tutorials/TBED07v00.htm}{Critical Editing}}.
\end{block}

\end{columns}

\framebreak

\begin{block}{Oxygen tricks}
\begin{itemize}\footnotesize
    \item If the TEI schema is linked to your document and you have internet, you can hover over elements and click to be redirected to the relevant info page. 
    \item If you open a tag (by just typing `\texttt{<}'), the editor will suggest a list of elements currently allowed where you're standing (for example, \texttt{<teiHeader>} is very picky about the sequence).
\end{itemize}
\end{block}

\end{frame}

%------------------------------------------------------------------------------
\begin{frame}[standout]
    \alert{TEI practice!} \\
    \begin{enumerate}\small
        \item Fill out the \texttt{teiHeader} or \texttt{msDesc}.
        \item Use websearch (`tei msDesc') to learn how to use new elements (\href{https://tei-c.org/release/doc/tei-p5-doc/en/html/ref-msDesc.html}{overview} plus \href{https://tei-c.org/release/doc/tei-p5-doc/en/html/examples-msDesc.html}{examples view}).
    \end{enumerate} 
\end{frame}




