
\usepackage[T1]{fontenc}
\usepackage{FiraSans} 


%
% Choose how your presentation looks.
%
% For more themes, color themes and font themes, see:
% http://deic.uab.es/~iblanes/beamer_gallery/index_by_theme.html
%
\mode<presentation>
{
  \usetheme[progressbar=foot,numbering=fraction,background=light]{metropolis}      % or try Darmstadt, Madrid, Warsaw, ...
  \usecolortheme{default} % or try albatross, beaver, crane, ...
  \usefonttheme{default}  % or try serif, structurebold, ...
  \setbeamertemplate{navigation symbols}{}
  \setbeamertemplate{caption}[numbered]
  %\setbeamertemplate{frame footer}{My custom footer}
} 

\usepackage[ngerman,english]{babel}
\usepackage{csquotes}
\usepackage[utf8x]{inputenc}


%--------------------------Editor mode.

\usepackage
[citestyle=authoryear,
%style=historian, 		%Loads the Historian files
sorting=nty,	  		%Sorts bibliography by year, name, title
autocite=footnote, 		%Autocite command generates footnotes
autolang=hyphen, 			%Allows hyphenation rules for foreign languages to apply to individual entries.						%(The other language rules should all be American)
mincrossrefs=1, 		%Includes all x-ref’ed entries in the bibliography
%usetranslator=true, 	%Translator’s name may be substituted for						%author or editor, if the latter are blank
%printseries, 			%Options provided by Historian, see below
backend=biber]
{biblatex}

\DeclareFieldFormat{postnote}{#1}
\DeclareFieldFormat{multipostnote}{#1}
\DeclareAutoCiteCommand{footnote}[f]{\footcite}{\footcites}

\bibliography{literature}
%----------------------------------------

%--------------------------Editor mode.


%\usepackage[default]{raleway}
\usepackage{fontawesome}


\usepackage{hyperref}
%\usepackage{enumitem} % produces fatal error with this template
%\setlist[itemize]{leftmargin=*} % belongs to enumitem

\usepackage{xcolor}
\definecolor{customcolor}{HTML}{616AC5}
\definecolor{alert}{HTML}{CD5C5C}
\definecolor{w3schools}{HTML}{4CAF50}
\definecolor{subbox}{gray}{0.60}
\definecolor{codecolor}{HTML}{FFC300}



 
%Information to be included in the title page:
\title[Beyond TEI] %optional
{\vspace{0.5cm} Beyond TEI}
\subtitle{Digital Editions with XPath \& XSLT for the Web \& in \LaTeX{}}
\institute{\raggedleft \includegraphics[height=1cm]{img/ninja-2000.png}\hspace{0.5cm}\includegraphics[height=1cm]{img/harvard-logo.png} }
\author[SL]{\raggedleft Sarah Lang}
\date[20220429] % (optional)
{\raggedleft Harvard, April/May 2022}

\logo{%\includegraphics[height=1cm]{unipassau.png}
%\includegraphics[height=1cm]{univie-logo.png}
%\includegraphics[height=2cm]{zim.png}
}



\newcommand{\punkti}{~\lbrack\dots\rbrack~}

\renewenvironment{quote}
               {\list{\faQuoteLeft\phantom{ }}{\rightmargin\leftmargin}%
                \item\relax\footnotesize\ignorespaces}
               {\unskip\unskip\phantom{xx}\faQuoteRight\endlist}

\newcommand{\bgupper}[3]{\colorbox{#1}{\color{#2}\huge\bfseries\MakeUppercase{#3}}}
\newcommand{\bg}[3]{\colorbox{#1}{\bfseries\color{#2}#3}}

\newcommand{\mycommand}[2]{{\ttfamily\detokenize{#1}}~\dotfill{}~{\footnotesize #2}\\}

\newcommand{\sep}{{\scriptsize~\faCircle{ }~}}

\newcommand{\red}[1]{\bg{alert}{white}{#1}\\}
\newcommand{\green}[1]{\bg{w3schools}{white}{#1}\\}




%----------------------------------------------------------------------------------------------------------------


\usepackage{xcolor}
\definecolor{customcolor}{HTML}{616AC5}
\definecolor{alert}{HTML}{CD5C5C}
\definecolor{w3schools}{HTML}{4CAF50}
\definecolor{subbox}{gray}{0.60}
\definecolor{codecolor}{HTML}{FFC300}


%--------------------------------------------------------------------------------
\usepackage{tcolorbox}

\tcbuselibrary{most,listingsutf8,minted}

\tcbset{tcbox width=auto,left=1mm,top=1mm,bottom=1mm,
right=1mm,boxsep=1mm,middle=1pt}

\newenvironment{mycolorbox}[2]{%
\begin{tcolorbox}[grow to left by=-1em,grow to right by=-1em,capture=minipage,fonttitle=\large\bfseries, enhanced jigsaw,boxsep=1mm,colback=#1!30!white,on line,tcbox width=auto, toptitle=0mm,colframe=#1,opacityback=0.7,nobeforeafter,title=#2]\scriptsize%
}{\end{tcolorbox}\\[0.2em]}

\newenvironment{subbox}[2]{%
\begin{tcolorbox}[capture=minipage,fonttitle=\normalsize\bfseries, enhanced jigsaw,boxsep=1mm,colback=#1!30!white,on line,tcbox width=auto,left=0.3em,top=1mm, toptitle=0mm,colframe=#1,opacityback=0.7,nobeforeafter,title=#2]\scriptsize %
}{\normalsize\end{tcolorbox}\vspace{0.1em}}

\newenvironment{multibox}[1]{%
\begin{tcbraster}[raster columns=#1,raster equal height,nobeforeafter,raster column skip=1em,raster left skip=1em,raster right skip=1em]}{\end{tcbraster}}


\newenvironment{mycodebox}[2]{%
\begin{tcolorbox}[grow to left by=-1em,grow to right by=-1em,capture=minipage,fonttitle=\large\bfseries, enhanced jigsaw,boxsep=1mm,colback=#1!30!white,on line,tcbox width=auto, toptitle=0mm,colframe=#1,opacityback=0.7,nobeforeafter,title=#2]%
}{\end{tcolorbox}\\[0.2em]}

\newtcolorbox{mybox}[2][]{colback=codecolor!10!white,coltitle=red!70!black,
title={#2},fonttitle=\bfseries,#1}


%-------------------------------

\newtcblisting{mypy}[1]{colback=codecolor!5,colframe=codecolor!80!black,listing only, 
minted options={numbers=left, style=tcblatex,fontsize=\scriptsize,breaklines,autogobble,linenos,numbersep=3mm},
left=5mm,enhanced,
title=#1, fonttitle=\bfseries,
listing engine=minted,minted language=python}

\newtcblisting{myxml}[1]{colback=codecolor!5,colframe=codecolor!80!black,listing only, 
minted options={numbers=left, style=tcblatex,fontsize=\tiny,breaklines,autogobble,linenos,numbersep=3mm},
left=5mm,enhanced,
title=#1, fonttitle=\bfseries,
listing engine=minted,minted language=xml}

\newtcblisting{mybiggerxml}[1]{colback=codecolor!5,colframe=codecolor!80!black,listing only, 
minted options={numbers=left, style=tcblatex,fontsize=\footnotesize,breaklines,autogobble,linenos,numbersep=3mm},
left=5mm,enhanced,
title=#1, fonttitle=\bfseries,
listing engine=minted,minted language=xml}

\newtcblisting{myhtml}[1]{colback=codecolor!5,colframe=codecolor!80!black,listing only, 
minted options={numbers=left, style=tcblatex,fontsize=\tiny,breaklines,autogobble,linenos,numbersep=3mm},
left=5mm,enhanced,
title=#1, fonttitle=\bfseries,
listing engine=minted,minted language=html}

\newtcblisting{mycss}[1]{colback=codecolor!5,colframe=codecolor!80!black,listing only, 
minted options={numbers=left, style=tcblatex,fontsize=\tiny,breaklines,autogobble,linenos,numbersep=3mm},
left=5mm,enhanced,
title=#1, fonttitle=\bfseries,
listing engine=minted,minted language=css}


\newtcblisting{myjs}[1]{colback=codecolor!5,colframe=codecolor!80!black,listing only, 
minted options={numbers=left, style=tcblatex,fontsize=\tiny,breaklines,autogobble,linenos,numbersep=3mm},
left=5mm,enhanced,
title=#1, fonttitle=\bfseries,
listing engine=minted,minted language=js}



%------------------------------------------------------



\definecolor{bgcolour}{rgb}{0.95,0.95,0.95}
\newminted{sql}{fontsize=\footnotesize, 
                   linenos,
                   %fontfamily=fi4, 
                   numbersep=6pt,
                   autogobble,
                   %frame=lines,
                   bgcolor=black!8,
                   framesep=3mm} 

%\usemintedstyle[shell-session]{vim} %  vim monokai fruity native
% color=black!80
\newminted{shell-session}{fontsize=\scriptsize,numbersep=6pt,bgcolor=black!70,autogobble,framesep=3mm%  %frame=lines, %fontfamily=fi4, 
                   }                    
                   
\newminted{sparql}{fontsize=\scriptsize, 
                   %linenos,
                   %fontfamily=fi4, 
                   numbersep=6pt,
                   autogobble,
                   %frame=lines,
                   bgcolor=black!8,
                   framesep=3mm} 
                   
\newminted{turtle}{fontsize=\scriptsize, 
                   %linenos,
                   %fontfamily=fi4, 
                   numbersep=6pt,
                   autogobble,
                   %frame=lines,
                   bgcolor=black!8,
                   framesep=3mm} 
                   
\newminted{js}{fontsize=\scriptsize, 
                   linenos,
                   %fontfamily=fi4, 
                   numbersep=6pt,
                   autogobble,
                   %frame=lines,
                   bgcolor=black!8,
                   framesep=3mm}                    

\newminted{xml}{fontsize=\scriptsize, 
                   %linenos,
                   %fontfamily=fi4, 
                   numbersep=6pt,
                   autogobble,
                   %frame=lines,
                   bgcolor=black!8,
                   framesep=3mm} 
                   
\newminted{tex}{fontsize=\scriptsize, 
                   %linenos,
                   %fontfamily=fi4, 
                   numbersep=6pt,
                   autogobble,
                   %frame=lines,
                   bgcolor=black!8,
                   framesep=3mm} 
                   
\newminted{postscript}{fontsize=\scriptsize, 
                   %linenos,
                   %fontfamily=fi4, 
                   numbersep=6pt,
                   autogobble,
                   %frame=lines,
                   bgcolor=black!8,
                   framesep=3mm} 
                   
\newminted{html}{fontsize=\scriptsize, 
                   %linenos,
                   %fontfamily=fi4, 
                   numbersep=6pt,
                   autogobble,
                   %frame=lines,
                   bgcolor=black!8,
                   framesep=3mm} 
                   
\newminted{css}{fontsize=\scriptsize, 
                   %linenos,
                   %fontfamily=fi4, 
                   numbersep=6pt,
                   autogobble,
                   %frame=lines,
                   bgcolor=black!8,
                   framesep=3mm} 
                   

%-------------------------------  
% https://tex.stackexchange.com/questions/377777/why-do-my-beamer-blocks-without-title-still-have-a-background
\usepackage{xstring}                
\setbeamertemplate{block begin}
{
  \par\vskip\medskipamount%
  \IfStrEq{\insertblocktitle}{}{}{
      \begin{beamercolorbox}[colsep*=.75ex]{block title}
        \usebeamerfont*{block title}\insertblocktitle%
      \end{beamercolorbox}%
  }
  {\parskip0pt\par}%
  \ifbeamercolorempty[bg]{block title}
  {}
  {\ifbeamercolorempty[bg]{block body}{}{\nointerlineskip\vskip-0.5pt}}%
  \usebeamerfont{block body}%
  \begin{beamercolorbox}[colsep*=.75ex,vmode]{block body}%
    \ifbeamercolorempty[bg]{block body}{\vskip-.25ex}{\vskip-.75ex}\vbox{}%
}
%-------------------------------  

